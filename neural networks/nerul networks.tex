\documentclass[UTF8]{ctexart}
\usepackage{geometry}
\usepackage{graphicx}
\usepackage{latexsym,bm,amsmath,amssymb}
\geometry{a4paper,scale=0.8}
\title{Neural Networks}
\author{AsukaShiKi}
\date{\today}
\begin{document}
\maketitle
\par
\begin{figure}[ht]
    \centering
    \includegraphics[scale=0.4]{01.jpg}
\end{figure}
\begin{figure}[ht]
    \centering
    \includegraphics[scale=0.3]{02.jpg}
\end{figure}
\newpage
\tableofcontents
\newpage
\Large{
\section{神经网络}
在机器学习中,神经网络一般是指“神经网络学习”。所谓神经网络,目前用的最广泛的一个定义是:“神经网络是由具
有适应性的简单单元组成的广泛并行互连的网络,它的组织能够模拟生物神经系统对真实世界物体所作出的反应。”,
他是一个黑箱模型,解释性较差,但是效果较好。
本章
}
\end{document}
